\chapter{Atividade 04}
\hypertarget{index}{}\label{index}\index{Atividade 04@{Atividade 04}}
\label{index_md__2_users_2renanoliveira_2_desktop_2ufscar_22023-02_2computacao-grafica_2atividades-projeto_2_atividade05_2_r_e_a_d_m_e}%
\Hypertarget{index_md__2_users_2renanoliveira_2_desktop_2ufscar_22023-02_2computacao-grafica_2atividades-projeto_2_atividade05_2_r_e_a_d_m_e}%
 {\bfseries{Descrição\+:}}

Este projeto consiste na implementação de um sistema de renderização de imagens usando ray tracing, seguindo o tutorial disponível em \href{https://raytracing.github.io/books/RayTracingInOneWeekend.html}{\texttt{ Ray Tracing in One Weekend}}. O objetivo principal é criar uma cena tridimensional e renderizá-\/la de dois pontos de vista diferentes, utilizando pelo menos três objetos com material difuso.

Para isso, o projeto utiliza conceitos de geometria computacional, álgebra linear e física da luz para simular o comportamento dos raios de luz ao interagirem com objetos na cena. O material difuso é implementado, utilizando a informação da normal do modelo para determinar a reflexão da luz em cada ponto da superfície.

Além disso, o projeto faz uso de arquivos contendo informações de normais dos modelos para criar uma representação mais realista das superfícies dos objetos. A cena é visualizada a partir de dois pontos de vista diferentes, proporcionando uma experiência de visualização abrangente e detalhada.

Ao implementar o material difuso e utilizar as informações de normal do modelo, o projeto busca criar uma cena visualmente atraente e realista, demonstrando os princípios fundamentais do ray tracing e oferecendo uma introdução prática à computação gráfica.

{\bfseries{Exemplo de Uso\+:}}


\begin{DoxyItemize}
\item adicionar Lib para salvar em PNG 
\begin{DoxyCode}{0}
\DoxyCodeLine{\$\ git\ clone\ https://github.com/lvandeve/lodepng.git}

\end{DoxyCode}

\item compilar e executar o código, as imagens serão salvas na pasta {\ttfamily /outputs} 
\begin{DoxyCode}{0}
\DoxyCodeLine{\$\ g++\ -\/std=c++11\ -\/o\ output\ main.cpp\ \ }
\DoxyCodeLine{\$\ ./output}

\end{DoxyCode}
 
\end{DoxyItemize}